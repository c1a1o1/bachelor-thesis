% $File: nasmia.tex
% $Date: Tue Mar 10 11:44:40 2015 +0800
% $Author: jiakai <jia.kai66@gmail.com>

\documentclass {beamer}
%\usetheme {JuanLesPins}
\usetheme{Malmoe}
%\usecolortheme{beaver}
\usepackage{fontspec,amsmath,amssymb,verbatim,mathtools,tikz}

% chinese
\usepackage{zhspacing}
\zhspacing
%\usepackage[noindent,UTF8]{ctexcap}

% biblatex
\usepackage{biblatex}
\bibliography{refs.bib}
\defbibheading{bibliography}{\section{}}
\setbeamertemplate{bibliography item}{\insertbiblabel}
\renewcommand*{\bibfont}{\footnotesize}

% minted
\usepackage{minted}
\newcommand{\inputmintedConfigured}[3][]{\inputminted[fontsize=\small,
	label=#3,linenos,frame=lines,framesep=0.8em,tabsize=4,#1]{#2}{#3}}
% \*src[additional minted options]{file path}
\newcommand{\cppsrc}[2][]{\inputmintedConfigured[#1]{cpp}{#2}}

% math opr
\DeclareMathOperator*{\argmin}{arg\,min}
\newcommand{\trans}[1]{#1^\intercal}

% others
\newcommand{\addgraph}[2]{\begin{center}
\includegraphics[width=#1\paperwidth]{#2}\end{center}}


\title{基于非监督深度学习的肝脏医学影像柔性配准算法研究}
\author {贾开}
\institute {清华大学}
\date{\today}

\begin{document}
\frame[plain]{\titlepage
    \begin{center}
        \begin{tabular}{ll}
            指导老师 & 宋亦旭 \\
            报告人 &  贾开 \\
            学号 & 2011011275
        \end{tabular}
    \end{center}
}

\AtBeginSection[]
{
    \begin{frame}<beamer>
        \frametitle{目录}
        \tableofcontents[currentsection]
    \end{frame}
}


\begin{frame}{目录}
    \tableofcontents
\end{frame}

\section{选题背景和意义}
% f{{{

\subsection{医学图像}

\begin{frame}{医学影像的自动分析和处理}
    \begin{itemize}
        \item 长期以来的研究热点,驱动机器视觉发展,
            但尚未完全解决
        \item 减轻医生负担,提高诊断准确率
        \item 交叉学科地带,
            整个领域随着医学、医学影像、计算机科学等领域的发展而不断变化
    \end{itemize}
    \addgraph{0.3}{res/mi-concept.jpg}
    \tiny{image source:
        \url{http://en.wikipedia.org/wiki/Medical\_imaging}}
\end{frame}


\begin{frame}{医学图像的配准}
    \begin{itemize}
        \item 病例到标准模板的匹配
        \item 术前术后的匹配
        \item 刚性配准(rigid registration) vs \\
            柔性配准(non-rigid
            registration/deformable registration)
    \end{itemize}
    \addgraph{0.5}{res/example-brain.png}
    \tiny{image source: \cite{shen2007image}}
\end{frame}


\subsection{深度学习}

\begin{frame}{神经网络的复兴--深度学习}
    \begin{itemize}
        \item 相关理论已在二十多年前成型,
            但并未展示出明显优势
        \item 三年前AlexNet血洗LSVRC2012千分类竞赛,
            再次将神经网络推向浪尖
        \item 深度学习已在机器视觉、机器翻译、自然语言处理、
            语音识别、文本识别等很多领域超越或远超传统方法
        \item 目前尝试将其用在医学图像配准的工作不多
    \end{itemize}

    \addgraph{0.5}{res/alexnet.png}
    \tiny{image source: \cite{krizhevsky2012imagenet}}
\end{frame}


\subsection{难点}
\begin{frame}{本研究面临的难点}
    \begin{description}
        \item[训练] 只有无标注的原始数据,只能使用非监督学习
        \item[测试] 目前我找到的数据集(LONI, IBSR, CUMC12, MGH10, NIREP, ADNI等)
            都是大脑影像,尚未发现肝脏或腹腔MR影像的数据集,
            因此定量测试上有一定困难
        \item[算法] 非刚性配准和深度学习各自有较成熟的发展,
            但公开发表的结合两者应用于医学影像的工作不多见,
            用于肝脏影像的工作我尚未发现,因此不确定最终能否work
    \end{description}
\end{frame}

% f}}}


\section{相关领域研究现状}
% f{{{

\subsection{非刚性配准}

\begin{frame}{问题描述}
    非刚性配准的问题可形式化表达如下\cite{sotiras2013deformable}:
    \begin{eqnarray*}
        W^* = \argmin_{W} M(T, S \circ W) + R(W)
    \end{eqnarray*}
    其中$W$为待求变换的参数表示,$S$和$T$分别是源影像与目标影像,
    $M(A, B)$评价$A$和$B$的不相似度,
    $R(W)$为$W$上基于先验知识的正则项。
\end{frame}

\begin{frame}{解法的关键因素}
    \begin{description}
        \item[变换模型] 物理模型、基于插值的模型、基于先验知识的模型、
            具体任务相关的模型等
        \item[匹配标准] 基于几何、基于图像、混合方法等
        \item[优化方法] 连续数值优化、离散优化、贪心法、进化算法等
    \end{description}
    该领域已有大量工作,
    本研究侧重于用非监督深度学习得到的特征值实现上述基于图像的匹配标准。
\end{frame}

\subsection{深度学习}

\begin{frame}{传说中的深度学习?}
    深度学习:人工神经网络的fancy别名
    \begin{description}
        \item[模型] 卷积神经网络(CNN),递归神经网络(RNN),受限玻耳兹曼机(RBM)等
        \item[优化方法] 随机梯度下降及其变种(Momentum, AdaGrad, rmsprop等)
        \item[tricks] Pretrain, Drop Out, ReLU, Maxout,
            Max/Average Pooling, Batch Normalization等等
    \end{description}
\end{frame}

\begin{frame}{为何深度学习可用?}
    LSVRC2012中,AlexNet取得前5猜想的15\%错误率,而最好的传统方法为26\%。
    \begin{description}
        \item[理论可行性] 足够大的神经网络可以无限逼近任意连续实函数
        \item[大数据] 相对二十年前,现在有了足够多的数据,使得大模型不会过拟合
        \item[训练技巧] 初始化、非线性、优化等各个环节的tricks
        \item[硬件设备] GPU使得快速大规模浮点运算成为可能
    \end{description}
\end{frame}

\begin{frame}{非监督的深度学习}
    \begin{description}
        \item[降噪自动编码器\cite{vincent2010stacked}]
            \hfill \\
            $h=\sigma(Wx+b), \tilde{x}=\sigma(\trans{W}h+c)$
        \item[层叠卷积自动编码器\cite{masci2011stacked}]
            \hfill \\
            $h=\sigma(x*W+b), \tilde{x}=\sigma(h*\tilde{W}+c)$
        \item[基于数据增广和判别式训练\cite{dosovitskiy2014discriminative}]
            \hfill \\
            采图像块并进行变换,要求从同一个图像块变换得到的图像块被分到一类
    \end{description}
\end{frame}

\subsection{直接相关的工作}
\begin{frame}{与肝脏医学影像配准直接相关的工作}
    \begin{itemize}
        \item 现有肝脏影像配准的相关工作不多,
            如\cite{carrillo2000semiautomatic}评测了十五年前的一些方法,
        \item 与本研究较接近的是\cite{wu2013unsupervised},
            使用非监督深度学习的方法进行大脑MR影像的配准,
            使用了两层的卷积ISA模型
        \item 现有工作尚未进入真正的临床应用,
            实际中是医生手标关键点求出刚性变换。
    \end{itemize}
\end{frame}

% f}}}

\section{本研究的内容和目标}

\subsection{本研究的内容和目标}

\begin{frame}{研究概览}
    \begin{description}
        \item[研究目标] \hfill \\
            尝试用非监督深度学习的方法得到更鲁棒、更具区分性的肝脏影像特征表示,
            并将该特征用于非刚性配准。
        \item[研究内容] \hfill \\
            \begin{itemize}
                \item 实现\cite{wu2013unsupervised}中算法并用于肝脏影像
                \item 尝试用前述的其他非监督深度学习方法习得特征,并比较效果
                \item 扩展内容:将上述特征用于半自动的病变区域选取
            \end{itemize}
        \item[可能的缺陷] \hfill \\
            本研究最大的可能缺陷是缺乏定量评测的带标注数据和评测算法,
            导致难以定量比较,可能只能靠人来直观感受。
    \end{description}
\end{frame}

\begin{frame}{基础算法概述}
    \addgraph{0.4}{res/isa.png}
    \begin{description}
        \item[特征提取\cite{wu2013unsupervised}]
            基于独立子空间分析(ISA),并将两层ISA层叠起来以处理高维输入的问题
        \item[配准模型\cite{shen2007image}]
            \tiny
            \begin{eqnarray*}
                h(u) &=& u + d(u) \\
                E &=& \sum_{u}
                \omega_T(u)\left(\frac{
                    \sum_{z\in n(u)}\varepsilon(z)(1-m(a_T(z), a_S(h(z))))}{
                    \sum_{z\in n(u)}\varepsilon(z)}\right) \\
                & & + \sum_{v}\omega_S(v)\left(\frac{
                    \sum_{z\in n(v)}\varepsilon(z)(1-m(a_T(h^{-1}(z)), a_S(z)))}{
                    \sum_{z\in n(v)}\varepsilon(z)}\right) \\
                & & + \beta\sum_u\Vert\nabla^2d(u)\Vert
            \end{eqnarray*}
    \end{description}
\end{frame}

\subsection{时间表}
\begin{frame}{时间表}
    \begin{tabular}{ll}
        寒假 & 研究数据格式,导出数据(已基本完成) \\
        1-3周 & 实现卷积ISA,得到基本可用的特征 \\
        4-7周 & 实现、调优并测试基本的配准算法 \\
        8周 & 准备期中检查 \\
        9-12周 & 尝试其他的非监督深度学习方法 \\
        13-15周 & 如果进展顺利,则尝试扩展内容 \\
        16周 & 整理实验内容,完成论文
    \end{tabular}
\end{frame}


\section{参考文献}
\begin{frame}[allowframebreaks]{参考文献}
    \printbibliography
\end{frame}


\section{ }
\subsection{ }
\frame{\begin{center}\huge{Thanks!}\end{center}}

\end{document}

% vim: filetype=tex foldmethod=marker foldmarker=f{{{,f}}} 

