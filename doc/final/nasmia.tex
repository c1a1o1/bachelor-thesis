% $File: nasmia.tex
% $Date: Mon Jun 22 14:54:21 2015 +0800
% $Author: jiakai <jia.kai66@gmail.com>

\documentclass {beamer}
%\usetheme {JuanLesPins}
\usetheme{Malmoe}
%\usecolortheme{beaver}
\usepackage{fontspec,amsmath,amssymb,verbatim,mathtools,tikz}

% chinese
\usepackage{zhspacing}
\zhspacing
%\usepackage[noindent,UTF8]{ctexcap}

% biblatex
\usepackage[citestyle=authoryear]{biblatex}
\bibliography{refs.bib}
\defbibheading{bibliography}{\section{}}
\setbeamertemplate{bibliography item}{\insertbiblabel}
\renewcommand*{\bibfont}{\footnotesize}

% math opr
../paper/mathdef.tex

% others
\newcommand{\addgraph}[2]{\begin{center}
\includegraphics[width=#1\paperwidth]{#2}\end{center}}
\newcommand{\addtwocolplot}[2]{\centering
    \begin{minipage}{0.49\textwidth}
        \centering
        \includegraphics[width=0.9\textwidth,
            height=0.4\paperheight,keepaspectratio]{#1}
    \end{minipage}
    \begin{minipage}{0.49\textwidth}
        \centering
        \includegraphics[width=0.9\textwidth,
            height=0.4\paperheight,keepaspectratio]{#2}
    \end{minipage}
}
\newcommand{\addplottcs}[1]{\begin{center}
    \includegraphics[width=0.441\textwidth,
    height=0.4\paperheight,keepaspectratio]{#1}
\end{center}}


\title{基于非监督深度学习的医学影像特征提取研究}
\author {贾开}
\institute {清华大学}
\date{\today}

\begin{document}
\frame[plain]{\titlepage
    \begin{center}
        \begin{tabular}{ll}
            指导老师 & 宋亦旭 \\
            报告人 &  贾开 \\
            学号 & 2011011275
        \end{tabular}
    \end{center}
}

\AtBeginSection[]
{
    \begin{frame}<beamer>
        \frametitle{目录}
        \tableofcontents[currentsection]
    \end{frame}
}


\begin{frame}{目录}
    \tableofcontents
\end{frame}

\section{研究背景与意义}
% f{{{

\begin{frame}{医学影像处理}
    意义重大,困难重重
    \begin{enumerate}
        \item 分辨率不高,
            如CT的图像矩阵往往只有$512\times 512$;
        \item 动态范围大,如CT值范围在$[-1000,1000]$Hu之间;
        \item 来源单一而封闭,需要大型扫描设备,而且涉及患者隐私;
        \item 由于人体构造、扫描时体位等不同而呈现很大差异性,
            病变部位则更是千差万别;
        \item 维度高,一般为3D体数据,或者带时间信息的4D图像。
    \end{enumerate}
    \addgraph{0.3}{res/mi-concept.jpg}
    \tiny{image source:
        \url{http://en.wikipedia.org/wiki/Medical\_imaging}}
\end{frame}

\begin{frame}{特征提取}
    好的特征:强\alert{鲁棒性}和高\alert{区分度}
    \begin{enumerate}
        \item 人工设计:LBP, HOG, SURF, SIFT等,性能有限
        \item 监督学习:大部分依赖于对匹配点的标注
        \item 非监督学习:大部分基于概率假设,
            对特征应具有的鲁棒性和区分力没有显式描述
    \end{enumerate}
\end{frame}

\begin{frame}{深度学习}
    关键点:
    \begin{enumerate}
        \item 数据
        \item GPU
        \item Tricks
    \end{enumerate}
    \pause
    \begin{itemize}
        \item 基于深度学习的方法已在机器视觉、机器翻译、自然语言处理、
            语音识别、文字识别等诸多领域取得领先结果
        \item 本文主要使用卷积神经网络
    \end{itemize}
    \addgraph{0.5}{res/alexnet.png}
    \tiny{image source: \cite{krizhevsky2012imagenet}}
\end{frame}

% f}}}

\section{研究内容}
% f{{{
\subsection{基于层叠卷积ISA的特征提取}
\begin{frame}{ISA:概览}
    \begin{itemize}
        \item 原始形式
        \item 层叠卷积ISA
        \item 实验
        \item 局限性
    \end{itemize}
\end{frame}

\begin{frame}{原始ISA的形式}
    划分子空间结构,仅要求子空间之间的独立性。
    $\vec{V}$为待求解的ISA矩阵,$\vec{z_1},\cdots,\vec{z_T}$
    为\alert{白化}的输入数据,
    $S(k)$为第$k$个子空间包含的下标,则优化目标为:
    \begin{eqnarray}
        \vec{V}^* &=& \argmin_{\vec{V}} \sum_{t=1}^T \sum_{k=1}^K
        \sqrt{\sum_{i\in S(k)}\left(\trans{\vec{v_i}}\vec{z_t}\right)^2}
        \nonumber \\
        \text{subject to} && \trans{\vec{V}}\vec{V}=\vec{I}
        \label{eqn:isa:1}
    \end{eqnarray}

    \only<1>{
        使用ISA从输入$\vec{x}$提取特征$\vec{e}$:
        \begin{eqnarray}
            e_k &=&  \sqrt{\sum_{i\in S(k)} s_i^2} \nonumber \\
            \vec{s} &=& \vec{V}\vec{x}
            \label{eqn:isa:extract}
        \end{eqnarray}
    }

    \only<2> {
        \vspace{1em}
        在$T$上拆分,多GPU\alert{并行实现}。
    }
\end{frame}

\begin{frame}{层叠卷积ISA}
    将4D图像作为输入,矩阵乘转化为卷积,成为两层神经网络
    \addgraph{0.8}{res/isa-stack.png}
    {\tiny image source: \cite{wu2013unsupervised}}
\end{frame}

\begin{frame}{实验: toy example}
    40维独立高斯分布:$\vec{x}=\trans{(x_1,\cdots,x_{40})}$ \\
    10维独立均匀分布:$\vec{y}=\trans{(y_1, \cdots, y_{10})}$
    \begin{eqnarray}
        \vec{z} &=& \trans{(z_1, \cdots, z_{40})} \nonumber \\
        z_{4i+j} &=& x_{4i+j+1}y_{i+1}
    \end{eqnarray}
    \addgraph{0.8}{res/isa-toyeg.eps}
\end{frame}

\begin{frame}{实验: SLIVER07上习得的滤波器}
    \addgraph{0.9}{res/isa-filter.png}
\end{frame}

\begin{frame}{局限性}
    \begin{enumerate}
        \item 卷积核大
        \item 贪心逐层训练
        \item 层数少
        \item 完全非监督
    \end{enumerate}
\end{frame}

\subsection{基于深度卷积神经网络的特征提取}

\begin{frame}{CNN:概览}
    \begin{itemize}
        \item 鲁棒性与区分力的形式化描述
        \item 两种损失函数
        \item 数据增广方法
    \end{itemize}
\end{frame}

\begin{frame}{鲁棒性与区分力}
    对于变换集合$\mathcal{T}$,定义
    \begin{eqnarray}
        \augdataset{X} &=& \left\{T(\vec{X}): T\in\mathcal{T}\right\}
            \nonumber \\
        \augftrset{X} &=& \left\{f(\vec{X};\vec{W}):
        \vec{X}\in\augdataset{X}\right\} \label{eqn:cnn:augdef}
    \end{eqnarray}

    要求:
    \begin{description}
        \item[鲁棒性] 同一位置的$\vec{X_1}$和$\vec{X_2}$,
            $\augftrset{X_1}$和$\augftrset{X_2}$类间变化\alert{小};
            $\vec{X}\in\mathcal{I}$,$\augftrset{X}$类内变化小
        \item[区分力] 不同位置的$\vec{X_1}$和$\vec{X_2}$,
            $\augftrset{X_1}$和$\augftrset{X_2}$类间变化\alert{大}
    \end{description}

\end{frame}

\begin{frame}{损失函数}
    \begin{description}
        \item[多分类]
            训练数据:$\vec{X_1},\cdots,\vec{X_N}$ \\
            监督信号:$\augdataset{X_i}$中的所有图像都被分类到第$i$类 \\
            \visible<3>{\alert{全局优化}}
            \pause
        \item[度量学习]
            不同位置的图像$\vec{X_1}, \vec{X_2}$ \\
            增广: $\vec{X_1}' \in \augdataset{X_1}, \vec{X_2}' \in \augdataset{X_1},
            \vec{X_3}' \in \augdataset{X_2}$ \\
            记$\vec{y_i}=f(\vec{X_i}';\vec{W})$
            \begin{eqnarray}
                L(\vec{y_1}, \vec{y_2}, \vec{y_3}) &=&
                    \max\left(0,\,
                        d(\vec{y_1}, \vec{y_2}) + \delta - d(\vec{y_1}, \vec{y_3})
                    \right) \nonumber \\
                d(\vec{x}, \vec{y}) &=& 1 -
                    \frac{\trans{\vec{x}}\vec{y}}{\abs{\vec{x}}\abs{\vec{y}}}
                \label{eqn:cnn:mtrc:loss}
            \end{eqnarray}
            \visible<3>{\alert{对数据量无苛刻要求}}
    \end{description}
\end{frame}

\begin{frame}{数据增广}
    数据增广本质上是定义$\mathcal{T}$的过程

    \begin{itemize}
        \item Gamma校正:
            \begin{eqnarray}
                \vec{y} &=& L + (U-L)\left(\frac{\vec{x}-L}{U-L}\right)^\gamma
            \end{eqnarray}
        \item 仿射变换
    \end{itemize}
\end{frame}

\begin{frame}{数据增广:仿射变换}
    对变换矩阵$\vec{A_0}$做SVD分解:
    \begin{eqnarray}
        \vec{A_0} &=&  \vec{U}\vec{\Sigma}\trans{\vec{V}} \nonumber \\
            &=& \vec{U}\vec{\Sigma}\trans{\vec{U}}\vec{R}
    \end{eqnarray}
    \addgraph{0.75}{res/affine-eg.png}
\end{frame}


\subsection{评测方法}
\begin{frame}{评测方法:概览}
    \begin{itemize}
        \item 单点匹配
        \item ROC曲线
    \end{itemize}
\end{frame}

\begin{frame}{单点匹配的判定}
    要求以前景掩膜形式提供对某内脏的分割标注
    \begin{description}
        \item[边界距离] 某点到标注上脏器边界的距离,脏器内部为正,外部为负
        \item[参考曲面] 边界距离为$2$的点构成
        \item[匹配点] 训练图像参考曲面上某点,
            到某测试图像所有点中特征距离最近的点
        \item[成功匹配] 匹配点在测试图像上的边界距离在$[1, 3]$范围内
    \end{description}
\end{frame}

\begin{frame}{成功匹配点对的抽样观察}
    \addgraph{0.8}{res/expr-match/pt-patch.png}

    \addgraph{0.8}{res/expr-match/pt-bd-dist.png}
\end{frame}

\begin{frame}{绘制ROC曲线}
    设有$N$个训练图像,每个的参考曲面上采样$T$个点;
    对测试图像$i$,对所有匹配点按$2\times 2\times 2$方块去重,
    然后限制特征距离不超过$\theta$:
    \begin{description}
        \item[顶峰比$t_i^{(\theta)}$]
            $\frac{\text{去重后点数}}{NT}$
        \item[精确度$a_i^{(\theta)}$]
            $\frac{\text{去重后成功匹配点数}}{\text{去重后点数}}$
    \end{description}
    $t^{(\theta)}$、$a^{(\theta)}$
    分别为
    $\{t_i^{(\theta)}\}$、$\{a_i^{(\theta)}\}$的均值,
    遍历$\theta$,绘制$(t^{(\theta)}, a^{(\theta)})$曲线。
\end{frame}

\begin{frame}{阈值限制对点分布的影响}
    所有匹配点,顶峰比为$0.568$,匹配精确度为$48.2\%$
    \addgraph{0.7}{res/expr-match/48.2-half.png}

    限制阈值,顶峰比为$0.078$,匹配精确度为$80.1\%$。
    \addgraph{0.7}{res/expr-match/80.1-half.png}
\end{frame}
% f}}}

\section{实验结果}
% f{{{

\begin{frame}{实验内容概览}
    \begin{itemize}
        \item 距离度量:cos vs $L_2$
        \item 训练数据量
        \item $\delta$ in $\max\left(0,\,
            d(\vec{y_1}, \vec{y_2}) + \delta - d(\vec{y_1}, \vec{y_3})\right)$
        \item PCA
        \item 边界距离与特征距离的关系反映特征性能
    \end{itemize}
\end{frame}

\subsection{ISA}
\begin{frame}{ISA}
    \addgraph{0.65}{res/expr/isa.pdf}
\end{frame}

\subsection{卷积神经网络}
\begin{frame}{距离度量}
    \addtwocolplot{res/expr/clsfy/measure/0.pdf}{res/expr/clsfy/measure/1.pdf}
    \addtwocolplot{res/expr/mtrc/measure/0-0.pdf}{res/expr/mtrc/measure/1-0.pdf}
\end{frame}

\begin{frame}{训练数据量}
    \addplottcs{res/expr/clsfy/datasize.pdf}
    \addtwocolplot{res/expr/mtrc/datasize/0.pdf}{res/expr/mtrc/datasize/1.pdf}
\end{frame}

\begin{frame}{度量学习中的$\delta$}
    \addtwocolplot{res/expr/mtrc/delta/0.pdf}{res/expr/mtrc/delta/1.pdf}
    \addplottcs{res/expr/mtrc/delta/2.pdf}
\end{frame}

\begin{frame}{PCA}
    \addtwocolplot{res/expr/clsfy/pca/0.pdf}{res/expr/clsfy/pca/1.pdf}
    \addtwocolplot{res/expr/mtrc/pca/0.pdf}{res/expr/mtrc/pca/1.pdf}
\end{frame}

\begin{frame}{边界距离、特征距离与特征性能}
    当小的边界距离不对应小的特征距离时,特征距离不反映匹配置信度,
    此时往往性能较差。如ISA的$L_2$、多分类时$5000$数据量等。

    \vspace{1em}
    \addtwocolplot{res/expr/border-dist-stat/l2.pdf}
        {res/expr/border-dist-stat/cos.pdf}
\end{frame}

\subsection{小结}
\begin{frame}{小结}
    \begin{enumerate}
        \item 监督信号加入对鲁棒性和区分力要求的CNN,比完全非监督的ISA好
        \item 每种方法有对应的恰当的距离度量
        \item 多分类训练的网络要求恰到好处的训练数据量
        \item 度量学习的损失函数中需要合适的$\delta$
        \item 结合PCA一般可提高性能
    \end{enumerate}
    最好结果:度量学习+大量数据+PCA+余弦距离
\end{frame}
% f}}}

\section{总结}
\begin{frame}{主要工作及创新点}
    \begin{enumerate}
        \item 层叠卷积ISA:
            GPU的数据并行实现,极大提高了训练速度
            \footnote{相关代码已开源
                \url{https://github.com/jia-kai/bachelor-thesis}}
        \item 深度卷积神经网络:将对鲁棒性和区分力的要求显式整合进损失函数,
            使用多分类和度量学习两种损失函数,辅以详细实验
        \item 用深度神经网络分析胸腹腔医学影像,并实验证明其在特征提取方面的优势
        \item 特征评测标准:无需依赖其它任务、无需对应点的标注
    \end{enumerate}
\end{frame}

\section{ }
\subsection{ }
\frame{\begin{center}\huge{Thanks!}\end{center}}

\section[]{参考文献}

\nocite{bay2006surf}
\nocite{bengio2009learning}
\nocite{bengio2013representation}
\nocite{bergstra+al:2010-scipy}
\nocite{dalal2005histograms}
\nocite{dosovitskiy2014discriminative}
\nocite{he2015delving}
\nocite{heimann2009comparison}
\nocite{hinton2015distilling}
\nocite{hornik1989multilayer}
\nocite{hyvarinen2000emergence}
\nocite{hyvarinen2009natural}
\nocite{kingma2014adam}
\nocite{krizhevsky2012imagenet}
\nocite{kuipers1999quaternions}
\nocite{le2011learning}
\nocite{lecun1998gradient}
\nocite{lowe1999object}
\nocite{medimging2}
\nocite{ojala1994performance}
\nocite{peyrat2010registration}
\nocite{schroff2015facenet}
\nocite{scovanner20073}
\nocite{shen2002hammer}
\nocite{sutskever2013importance}
\nocite{wu2013unsupervised}

\begin{frame}[allowframebreaks]{参考文献}
    \printbibliography
\end{frame}

\end{document}

% vim: filetype=tex foldmethod=marker foldmarker=f{{{,f}}}

