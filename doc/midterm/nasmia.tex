% $File: nasmia.tex
% $Date: Mon Apr 20 22:58:50 2015 +0800
% $Author: jiakai <jia.kai66@gmail.com>

\documentclass {beamer}
%\usetheme {JuanLesPins}
\usetheme{Malmoe}
%\usecolortheme{beaver}
\usepackage{fontspec,amsmath,amssymb,verbatim,mathtools,tikz}

% chinese
\usepackage{zhspacing}
\zhspacing
%\usepackage[noindent,UTF8]{ctexcap}

% biblatex
\usepackage{biblatex}
\bibliography{refs.bib}
\defbibheading{bibliography}{\section{}}
\setbeamertemplate{bibliography item}{\insertbiblabel}
\renewcommand*{\bibfont}{\footnotesize}

% math opr
\DeclareMathOperator*{\argmin}{arg\,min}
\newcommand{\trans}[1]{#1^\intercal}
\renewcommand{\vec}[1]{\boldsymbol{#1}}

% others
\newcommand{\addgraph}[2]{\begin{center}
\includegraphics[width=#1\paperwidth]{#2}\end{center}}


\title{基于深度学习的医学影像配准算法研究}
\subtitle{中期检查}
\author {贾开}
\institute {清华大学}
\date{\today}

\begin{document}
\frame[plain]{\titlepage
    \begin{center}
        \begin{tabular}{ll}
            指导老师 & 宋亦旭 \\
            报告人 &  贾开 \\
            学号 & 2011011275
        \end{tabular}
    \end{center}
}

\AtBeginSection[]
{
    \begin{frame}<beamer>
        \frametitle{Table of Contents}
        \tableofcontents[currentsection]
    \end{frame}
}


\begin{frame}{Table of Contents}
    \tableofcontents
\end{frame}

\section{Current Progress}

\subsection{The Model: ISA}
% f{{{
\begin{frame}{Motivation}
    \begin{description}
        \item[ISA] Independent Subspace Analysis
        \item[Motivation] Maximize independence of features, while allowing
            dependence inside subspaces with predefined structures.
        \item[Pre-requirement] whitened data
    \end{description}
\end{frame}

\begin{frame}{Interpretation 1}
    Maximizing likelihood
    \begin{eqnarray*}
        \vec{s} &=& \vec{Wx} \\
        u_i &=& \sqrt{\sum_{j \in S_i}s_j^2} \\
        p(\vec{s}) &=& \prod_i{\frac{1}{Z_i}\exp\left(-\frac{u_i}{b}\right)} \\
        \argmin_{\vec{W}} && \text{E}_{\vec{x}}\left[\sum_i u_i\right] \\
        \text{subject to} && \vec{W}\trans{\vec{W}} = \vec{I}
    \end{eqnarray*}
\end{frame}

\begin{frame}{Interpretation 2}
    As a 2-layer neural network
    \addgraph{0.8}{res/isa.png}
    {\tiny image from \cite{le2011learning}}
\end{frame}

\begin{frame}{Interpretation 3}
    Sparse coding with orthonormal regularizer.
    \begin{eqnarray*}
        \argmin_{\vec{W}} && \text{E}_{\vec{x}}\left[\sum_i u_i\right] \\
        \text{subject to} && \vec{W}\trans{\vec{W}} = \vec{I}
    \end{eqnarray*}
\end{frame}

\begin{frame}{Stacked ISA}
    \addgraph{0.6}{res/isa-stack.png}
    {\tiny image from \cite{wu2013unsupervised}}
\end{frame}
% f}}}

\subsection{Optimization}
% f{{{
\begin{frame}{Optimization Method}
    Batched gradient decent:
    \begin{eqnarray*}
        \vec{W} \leftarrow \vec{W} - \alpha
        \frac{1}{N}\sum_{i=1}^N \frac{\partial{T(\vec{x_i};\vec{W})}}
            {\partial{\vec{W}}}
    \end{eqnarray*}
\end{frame}

\begin{frame}{Implementation Details}
    \begin{description}
        \item[Utilities] python2 + theano
        \item[Parallel Computing] Data parallel on GPUs, both PCA and batched
            gradient computing/update are parallelized.
        \item[Network Structure]
            \begin{tabular}{llll}
                Layer & Input Patch Size & Hidden Ftr & Output Ftr \\
                1 & $13\times13\times13$ & 300 & 150 \\
                2 & $21\times21\times21$ & 200 & 50
            \end{tabular}
            Note: layer 1 convolution stride is 8, so actual layer 2 input size
            is $150 \times 2 \times 2 \times 2$.
        \item[Hardware]
            Intel(R) Core(TM) i7-4930K CPU @ 3.40GHz,
            55GiB RAM, 4 Nvidia Titan GPUs
        \item[Efficiency] For layer 2: 5.5GiB training data, 15secs for PCA,
            0.15secs for each iteration.
    \end{description}
\end{frame}
% f}}}

\subsection{Results}
% f{{{
\begin{frame}{Filters of Layer 1}
    \addgraph{0.6}{res/filter.png}
\end{frame}

\begin{frame}{Feature Discriminability and Robustness}
    \addgraph{0.7}{res/dist.png}
\end{frame}
% f}}}

\section{Future Work}
\begin{frame}{Problems and Future Work}
    \begin{description}
        \item[Major Problem] No handy tool for non-rigid registration with
            custom features
        \item[Future Work]
            \begin{enumerate}
                \item Try other feature learning methods
                \item Test feature performance by rigid registration using
                    RANSAC.
            \end{enumerate}
    \end{description}
\end{frame}


\section{ }
\subsection{ }
\frame{\begin{center}\huge{Thanks!}\end{center}}

\section[]{References}
\begin{frame}[allowframebreaks]{References}
    \printbibliography
\end{frame}

\end{document}

% vim: filetype=tex foldmethod=marker foldmarker=f{{{,f}}}

