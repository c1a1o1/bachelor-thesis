% $File: ISA.tex
% $Date: Mon Jun 01 23:53:04 2015 +0800
% $Author: jiakai <jia.kai66@gmail.com>

\chapter{基于层叠卷积ISA的特征提取}
本章主要介绍独立子空间分析(Independent Subspace Analysis, ISA)方法的基本原理,
及将其多次层叠后构成深度网络进行特征提取的方法,并对该方法进行简单的讨论。

\section{ISA的基本原理}
ISA是对独立成分分析(Independent Component Analysis, ICA)的扩展,
是一种经典的统计学习方法。
在本节中,先对ICA进行介绍,再将其扩展到ISA。关于ISA的更为详细的内容,
可以参考\cite{hyvarinen2009natural}。


\subsection{ICA的基本原理}
% f{{{
ICA是一种生成模型,其出发点是希望从对一个随机变量的一系列观察中,
分析出其背后的独立成分,每个成分有自己的概率分布,从而得出该随机变量的概率分布。

具体而言,对随机变量$\vec{x} = \trans{(x_1, \cdots, x_n)}$,ICA假设
\begin{eqnarray}
    \vec{x} = \vec{A}\vec{s}
    \label{eqn:ica:0}
\end{eqnarray}
其中,$\vec{A}$是一个$n\times m$的矩阵,$\vec{s}=\trans{(s_1,\cdots,s_m)}$
是一个$m$维随机变量,对于$i\neq j$,$s_i$和$s_j$独立。

实际应用时,一般$m \le n$,先对$\vec{x}$进行主成分分析(Principal Component
Analysis, PCA),降维成$m$维随机变量$\vec{z}=\vec{P}(\vec{x}-\bar{\vec{x}})$,
其中$\vec{P}$是$m \times n$的PCA矩阵;
另外\eqnref{ica:0}中的$\vec{x}$对应替换成$\vec{z}$,即
\begin{eqnarray}
    \vec{z} = \vec{B}\vec{s}
    \label{eqn:ica:1}
\end{eqnarray}
其中$\vec{B}$是$n\times n$矩阵,
用于将假设的独立特征空间变换到观察到的随机向量空间。显然\eqnref{ica:1}可逆,
记$\vec{B}^{-1}=\vec{V}$,有
\begin{eqnarray}
    \vec{s} = \vec{V}\vec{z}
    \label{eqn:ica:2}
\end{eqnarray}

在\eqnref{ica:0}中,$\vec{s}$解释了$\vec{x}$的概率分布,
可认为是一种更易处理、更能反应$\vec{x}$本质的特征;
而\eqnref{ica:2}则给出了从$\vec{x}$提取特征$\vec{s}$的方法,
其中$\vec{V}$就是特征检测器。

下面将简述基于最大似然来求解$\vec{V}$的方法。基于独立性的假设,有
\begin{eqnarray}
    \prob{\vec{s}} = \prod_{i=1}^m\prob[i]{s_i}
    \label{eqn:ica:3}
\end{eqnarray}
联合\eqnref{ica:2},于是有
\begin{eqnarray}
    \prob{\vec{z}} &=&  \prob{\vec{V}^{-1}\vec{s}} \nonumber \\
        &=& \abs{\det{\vec{V}}} \prob{\vec{s}} \nonumber \\
        &=& \abs{\det{\vec{V}}}\prod_{i=1}^m\prob[i]{s_i} \nonumber \\
        &=& \abs{\det{\vec{V}}}\prod_{i=1}^m\prob[i]{\trans{\vec{v_i}}\vec{z}}
    \label{eqn:ica:4}
\end{eqnarray}

假设对于$\vec{z}$有$T$个独立观测结果$\vec{z_1}\cdots\vec{z_T}$,
可定义最大似然为优化目标,于是ICA的参数$\vec{V}$估计如下:
\begin{eqnarray}
    \vec{V}^* &=& \argmax_{\vec{V}} L(\vec{V}) \nonumber \\
    &=& \argmax_{\vec{V}} \prod_{t=1}^T \prob{\vec{z_t}|\vec{V}} \nonumber \\
    &=& \argmax_{\vec{V}} \prod_{t=1}^T \left(\abs{\det{\vec{V}}}
            \prod_{i=1}^m\prob[i]{\trans{\vec{v_i}}\vec{z_t}} \right)
    \label{eqn:ica:5}
\end{eqnarray}

另外,由独立性,还应该要求$\vec{s}$的各分量不相关,即$\cov{\vec{s}} = \vec{I}$,
则需要$\vec{V}$满足$\trans{\vec{V}}\vec{V}=\vec{I}$,
于是有$\abs{\det{\vec{V}}}=1$,ICA的求解最终为如下形式:
\begin{eqnarray}
    \vec{V}^* &=& \argmax_{\vec{V}}
            \prod_{t=1}^T \left(
            \prod_{i=1}^m\prob[i]{\trans{\vec{v_i}}\vec{z_t}} \right) \nonumber
            \\
        \text{subject to} && \trans{\vec{V}}\vec{V}=\vec{I}
    \label{eqn:ica:6}
\end{eqnarray}

% f}}}

\subsection{扩展ICA到ISA}
% f{{{
从\eqnref{ica:2}得到的特征$\vec{s}$本质上是原始输入$\vec{x}$的线性变换。
线性变换的一个不足便是无法表达不变性,即输入发生任何改变,输出也都会对应变化,
无法对某些变换(如图像的小范围平移、旋转等)保持结果的稳定。

因此对ICA进行如下改进得到ISA:将特征$\vec{s}$分为$K$组,
第$k$组对应的分量下标集合记作$S(k)$。把每组特征看作一个子空间,
将其能量作为该子空间的特征输出。具体而言:
\begin{eqnarray}
    e_k = \sqrt{\sum_{i\in S(k)} s_i^2}
    \label{eqn:isa:0}
\end{eqnarray}

$\vec{e} = \trans{(e_1,\cdots,e_K)}$就是ISA最终的特征输出。

在ISA中,对$\vec{s}$各分量的独立性不做假设,而是假设$\vec{e}$的分量间独立,
并且仍然要求$\trans{\vec{V}}\vec{V}=\vec{I}$以得到尽量丰富的特征。
类似\eqnref{ica:6},基于最大似然对$\vec{V}$求解:
\begin{eqnarray}
    \vec{V}^* &=& \argmax_{\vec{V}}
            \prod_{t=1}^T \left(
            \prod_{k=1}^K\prob[k]{e_k} \right) \nonumber \\
        &=& \argmax_{\vec{V}}
            \prod_{t=1}^T \left(
            \prod_{k=1}^K\prob[k]{\sqrt{
                \sum_{i\in S(k)}\left(\trans{\vec{v_i}}\vec{z_t}\right)^2}
            }\right) \nonumber \\
        \text{subject to} && \trans{\vec{V}}\vec{V}=\vec{I}
    \label{eqn:isa:1}
\end{eqnarray}

一般而言,$p_k(s)$取为拉普拉斯分布(Laplace distribution),
即$p_k(s) = \frac{1}{2b}\exp(-\frac{\abs{s}}{b})$,
并在优化目标上取对数,最终ISA参数求解的形式如下:
\begin{eqnarray}
    L(V) &=& \sum_{t=1}^T \sum_{k=1}^K
        \sqrt{\sum_{i\in S(k)}\left(\trans{\vec{v_i}}\vec{z_t}\right)^2}
        \nonumber \\
    \vec{V}^* &=& \argmin_{\vec{V}} L(V) \nonumber \\
        \text{subject to} && \trans{\vec{V}}\vec{V}=\vec{I}
    \label{eqn:isa:opt}
\end{eqnarray}

而使用ISA提取特征的方法如下:
\begin{eqnarray}
    e_k &=&  \sqrt{\sum_{i\in S(k)} s_i^2} \nonumber \\
    \vec{s} &=& \vec{V}\vec{P}(\vec{x} - \bar{\vec{x}})
    \label{eqn:isa:extract}
\end{eqnarray}

% f}}}

\section{层叠卷积ISA}
% f{{{
以3D的CT扫描图像为例,要将ISA应用于图像处理,
一般是在扫描结果的数据里随机抽取$T$个$p\times p \times p$的图像小块,
把每个图像块平整化,看作一个$p^3$维的向量,将这些向量带入\eqnref{isa:opt}
求解参数$\vec{V}$。

处理图像数据时,ISA可以被放入卷积神经网络的框架中。
在\eqnref{isa:extract}中,
记$\vec{A} = \vec{V}\vec{P}$,$\vec{b}=-\vec{A}\bar{\vec{x}}$,
则$\vec{s}=\vec{A}\vec{x} + \vec{b}$可以看作一个全连接隐层输出。
记$\vec{A}$的维度是$m\times n$,而当$n=p^3$、
$\vec{x}$对应于一个$p\times p \times p$的图像块时,
$\vec{A}$可对应看作一个$m\times 1 \times p \times p \times p$的卷积核,
而其后的子空间的能量响应则可以看作是在多通道3D图像上进行的跨通道的非线性操作,
从而单层ISA可以看作由卷积以及非线性操作组成的卷积神经网络,
如\figref{isa:nn}所示。
在这种框架下,单层的ISA可以被用于任意大小的输入图像上,
在图像上的每个点密集提取特征;在GPU的辅助下可以达到很高的速度。

\begin{figure}[h!]
    \addplot{res/placeholder.png}
    \caption{单层ISA对应的卷积神经网络结构}
    \label{fig:isa:nn}
\end{figure}

但上述标准的单层ISA有两大缺点:
\begin{enumerate}
    \item 只有一层非线性,无法表达更高层更复杂的结构特征
    \item 由于优化过程中需要不断将权重矩阵正规化为正交阵,
        该过程复杂度为$\Theta(n^3)$,因此当输入图像块的维度较大时,
        训练会非常耗时。
\end{enumerate}

为了解决这两个问题,可以将多个这样的卷积ISA层叠起来,
从而作为一个深度特征提取器来使用。训练时可采取贪心逐层训练的方法,
在训练底层时使用较小的图像块,随后转换为卷积形式,
用较大的图像块作为输入并提取特征,在提取出的特征上再训练下一层ISA。

% f}}}

\section{训练方法及其实现}
% f{{{
为了优化\eqnref{isa:opt},我们使用带投影的整批梯度下降的方法。
先从均匀分布中随机采样得到初始权重矩阵$\vec{V_0}$;随后按如下规则迭代更新:
\begin{eqnarray}
    \vec{W_i} &=& \vec{V_{i-1}} - \alpha \frac{\partial L}{\partial
        \vec{V}}(\vec{V_{i-1}}) \nonumber  \\
        \vec{V_i} &=& \left(
            \vec{W_i}\trans{\vec{W_i}}\right)^{-\frac{1}{2}}\vec{W_i}
    \label{eqn:isa:train}
\end{eqnarray}
其中$\vec{W_i}$是沿梯度方向更新后的权重矩阵,而$\vec{V_i}$则是将$\vec{W_i}$
正规化使其成为正交阵。$\alpha$为学习速率,需要调整到合适的值使得$L(\vec{V_i})$
能较快下降而又不至于发生不稳定震荡。训练直到$L(\vec{V_i})$收敛才停止。

在实现方面,我们使用了theano\cite{bergstra+al:2010-scipy}作为训练框架,
用python实现训练功能。theano是一个符号计算框架,支持自动求导,
可以透明地实现GPU和CPU计算后端切换,同时内置了矩阵乘法、
卷积等各种常见操作的高效实现。

为了加速训练,我们实现了数据并行,可同时利用同一台主机上的多个GPU一起计算。
观察\eqnref{isa:opt}的损失函数$L(V)$,及\eqnref{isa:train}的更新规则,
可以发现在计算$\vec{W_i}$时很容易进行数据并行,
只需要将$T$个训练样本拆分成$N$份,各自求出的梯度相加后即可得到总体梯度,
然后再在CPU上更新$\vec{W_i}$及正规化得到$\vec{V_i}$。
这样单层的实际训练时间可以在半小时以内。


\subsection{对实现的简单验证}
在本小节中,我们通过简单的合成数据,来对ISA实现的正确性进行验证。
考虑各维服从独立高斯分布的40维随机变量$\vec{x}=\trans{(x_1,\cdots,x_{40})}$,
和各维服从独立均匀分布的10维随机变量$\vec{y}=\trans{(y_1, \cdots, y_{10})}$,
令
\begin{eqnarray}
    \vec{z} &=& \trans{(z_1, \cdots, z_{40})} \nonumber \\
    z_{4i+j} &=& x_{4i+j+1}y_{i+1} \nonumber \\
    && \forall 0 \le i \le 9,\,0 \le j \le 3
\end{eqnarray}

这样定义$\vec{z}$有两大好处,
一方面$\vec{z}$满足了ISA所需要的super-Gaussian分布,
另一方面将$\vec{z}$的各分量每4个分一组,则组内分量间也有了依赖性,
可以测试ISA的性能。采样$10000$个$\vec{z}$得到$40\times 10000$的矩阵$\vec{Z}$,
然后再从均匀分布中采样得到一个$64\times 40$的矩阵$\vec{M}$作为混合矩阵,
用$\vec{I} = \vec{M}\vec{Z}$作为最终呈现给ISA算法的输入。
为了评价ISA的效果,在求得ISA的权重矩阵$\vec{V}$后,
带入\eqnref{ica:2}中,然后计算$\corr{s_i^2, s_j^2}$作为$(i, j)$
处的元素绘制在\figref{isa:test}中,可以看出ISA能还原出这些随机变量间的内在结构,
被分在同一组的变量间也有较高的相关性。

\begin{figure}[h!]
    \addplot{res/isa-toyeg.eps}
    \caption{在合成数据上ISA还原出的输入变量的内在关系}
    \label{fig:isa:test}
\end{figure}

% f}}}

\section{实验配置\label{sec:isa:expr}}
% f{{{
在本文中,均使用两层卷积ISA。第一层训练时输入的图像块大小为
$13\times 13 \times 13$,先用PCA把数据降维到$600$维,
每个子空间包含$2$个线性特征,输出维度为$300$维;
第二层的原始输入图像块大小为$21 \times 21 \times 21$,
先在图像块上以$8$为步长、用第一层卷积ISA得到$300$通道的$2\times 2 \times 2$
的图像块,共$2400$维作为第二层ISA训练的输入向量,PCA降维至$200$维,
子空间大小为$4$,最终输出特征维度为$50$维。

在SLIVER07的腹腔CT扫描数据上习得的第一层特征检测器如\figref{isa:filter}所示。
观察可以发现,其中很多都像检测各种朝向边缘的Gabor filter;
相邻两个检测器属于同一个子空间,可以看到它们的形态相似而在相位上有区别,
这也表明了一个子空间所对应的特征具有一定的平移不变性。

\begin{figure}[h!]
    {
        \addplot{res/isa-filter.png}
        \caption{在实际数据上用ISA习得的第一层特征检测器}
        \label{fig:isa:filter}
    }
    \footnotesize
    图中展示了每个输出通道所对应的检测器,由于检测器本身是3D的,
    这里为了展示方便,仅选取了其中某一维的中间面片。
\end{figure}

% f}}}

\section{小结与讨论\label{sec:ISA:discuss}}
% f{{{
本节主要对ICA和ISA进行了介绍,描述了ISA到卷积神经网络的转化,并简述了训练方法。

ISA作为一种非监督学习的方法,其习得的特征可以自发的展现一定的平移和相位不变性
\cite{hyvarinen2000emergence}。
层叠卷积ISA被应用在了视频中的动作识别\cite{le2011learning}、
人脑MRI扫描图像的配准\cite{wu2013unsupervised}等领域,均取得不错的结果。

然而,叠卷积ISA也有一定的局限性:
\begin{enumerate}
    \item 卷积核大小需要与输入图像块的大小相同,使得对应的卷积核较大,
        而较大的卷积核会导致较慢的运行时间与较多的内存占用;
    \item 训练时采取贪心逐层训练的方法,缺乏全局优化的过程;
    \item 层数少,无法表达更复杂的结构,
        而且单层ISA的训练也要求输入向量维度不能太少,
        因此要想构造深层网络就需要很大的输入图像块;
    \item 最主要的一个缺陷是,ISA是一种完全非监督的方法;
        而实际应用中,我们往往希望特征对一定的平移、旋转等扰动具有不变性,
        这种要求无法整合进ISA的框架。
        虽然在实验中人们发现ISA习得的特征具有一定平移和相位不变性,
        可是对这种不变性的形式和程度都没有理论保障。
\end{enumerate}
% f}}}

% vim: filetype=tex foldmethod=marker foldmarker=f{{{,f}}}
