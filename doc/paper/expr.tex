% $File: expr.tex
% $Date: Sun Jun 07 23:50:02 2015 +0800
% $Author: jiakai <jia.kai66@gmail.com>

\chapter{评测方法与实验结果}
本章首先提出一种评测方法,其完全基于肝脏分割标注,
无需基于其它的柔性匹配算法,也无需显式计算两个标注间的点对应关系,
可以直接评价特征的优劣,简便高效。随后本章介绍具体实验配置,
并报告基于该评测方法的实验结果。

\section{评测方法}
在实际应用中,往往难以有单一可靠的方法直接评判特征优劣,
而是需要有一个依赖某特征的具体任务,
并通过该特征在该任务上的表现来间接反应特征优劣。
例如,SIFT特征最早被用于物体识别\cite{lowe1999object},
层叠卷积ISA被用于动作识别\cite{le2011learning}、
大脑MRI扫描的柔性配准\cite{wu2013unsupervised}等。
然而,如果任务过于复杂,
结果往往会受到任务相关的具体算法的影响。
例如在Guorong Wu的工作\cite{wu2013unsupervised}中,
作者的评测过程依赖于第三方软件的预处理,
而且发现换用ISA特征后在Demons算法上的配准性能反而变差了,
虽然作者表示这是由于实验过程带来的一些不公平造成的,
但客观而言在这种复杂的环境下确实更难分离出特征本身在最终性能里的贡献。
因此,在本节中,我们将提出一个新的简单而普适的方法来评测特征性能,
以免结果受到过于复杂的任务相关算法的影响。
简单而言,该方法基于人工标定的器官分割掩膜,在不需要具体点对应关系的情况下,
来评价一个特征在测试图像里寻找参考图像中某个点的准确度。

\subsection{基于器官分割标注的曲面匹配}
我们的评测方法要求数据提供对某个器官的分割标注,
例如在本文中我们使用SLIVER07中的肝脏分割标注。
在本小节中,我们将介绍基于大量带器官分割标注的训练数据,
在单个测试图像上寻找某个曲面的方法。为此,我们先定义单点匹配,
并将其扩展到曲面。

\subsubsection{单点匹配}
基于分割标注,我们对每个点都定义{\bf 边界距离},
并基于边界距离来判断单点是否匹配成功。

我们假设器官分割掩膜以二值3D图像$\vec{M}$的形式提供,
$\vec{M}$中某点值为$1$是表示对应点属于目标器官,为$0$时表示不属于目标器官。
对于每个点$(i, j, k)$,定义
\begin{eqnarray}
    N(i, j, k) &=& \prod_{
        \max(\abs{x}, \abs{y}, \abs{z}) = 1}
        M(i + x, j + y, k + z)
\end{eqnarray}
$N(i, j, k)$表示了与$(i, j, k)$相邻的点中是否有不属于目标器官的点。
于是可定义边界点集为:
\begin{eqnarray}
    \partial M &=& \left\{\,\vec{p} : M(\vec{p}) = 1\,\text{且}\,
        N(\vec{p}) = 0\,\right\}
\end{eqnarray}

把每个点看作无向图的顶点,同时在几何上看作一个单位立方体,
对于有公共顶点或公共边的两点间连一条权值为$1$的边,
于是对任意两点$\vec{p}, \vec{q}$,其间存在最短路,
距离记作$s(\vec{p}, \vec{q})$。
对每个点$\vec{p}$,可定义无符号边界距离$D(\vec{p})$和边界距离$d(\vec{p})$:
\begin{eqnarray}
    D(\vec{p}) &=& \min_{\vec{q} \in \partial M} s(\vec{p}, \vec{q}) \\
    d(\vec{p}) &=& (2M(\vec{p})-1)D(\vec{p})
\end{eqnarray}
在涉及多个图像如$\vec{M_1}$、$\vec{M_2}$时,
我们通过脚注形式来区分各自的边界距离:
$d_{\vec{M_1}}(\vec{p})$、$d_{\vec{M_2}}(\vec{q})$。

对于参考图像$\vec{R}$上的某点$\vec{p}$及其特征$\vec{f(p)}$,
我们在测试图像$\vec{T}$上寻找特征距离最小的点$\vec{q}$,
称$\vec{q}$为$\vec{p}$的匹配点,
如果还有$\abs{d_{\vec{R}}(\vec{p}) - d_{\vec{T}}(\vec{q})} \le \theta$,
则认为匹配成功,其中$\theta$为容忍的距离误差,本文中均取$1$。
为了在特征上快速、精确地寻找匹配点,即特征空间上的最近邻,
我们把两两特征间的距离计算转换成矩阵乘法并在GPU上运行。

\subsubsection{曲面匹配\label{sec:expr:match}}
上述单点匹配的判别方法,易受各种随机因素的影响,
在这里我们将其扩展到曲面匹配以提高鲁棒性。

首先根据边界距离,对器官分割标注$\vec{M}$定义参考曲面:
\begin{eqnarray}
    \hat{\vec{M}} &=& \left\{ \vec{p} : d_{\vec{M}}(\vec{p}) = d_0 \right\}
\end{eqnarray}
在本文中,取参考曲面为边界稍靠内的曲面,取$d_0=2$,
这样的一个好处是所有可匹配点的距离在$[1, 3]$间,也都在目标器官内部。

假设我们有$N$个训练数据$\vec{M_1},\cdots,\vec{M_N}$,
我们在$\hat{\vec{M_1}},\cdots,\hat{\vec{M_N}}$上各均匀选取$T$个点,
并在测试图像的上寻找这$NT$个点的匹配点。
为了防止特征只注重了很明显的局部特点而导致匹配点过于集中,
我们把测试图像分成了若干个小方格,每个的大小为$k\times k \times k$,
对于落入同一方格的匹配点,只记录其中距离最小的点,
匹配精确度定义为这些剩下的点中成功匹配的点数占剩下的点总数的比例。
在本文的实验中,均取$T=3000, k=2$。

\figref{expr:match}中给出了在此评测标准下,
达到$63.5\%$准确率的实际匹配点的分布,可以看出总体来说还是符合预期的。

\begin{figure}[h!]
    {
        \addplot{res/expr-match.png}
        \caption{$63.5\%$准确率下实际匹配点的分布}
        \label{fig:expr:match}
    }
    \footnotesize
    该图通过在x轴上按4像素为步长切片绘制。
    其中彩色标注的区域是人工标注的肝脏区域,绿色点为成功匹配点,
    蓝色点为失败匹配点,蓝色、绿色颜色越亮,则表示匹配的特征距离越小。
    为方便查看,每个点用原始点为中心的$5\times 5 \times 5$立方体来表示,
    因此在某些切片上的一些成功匹配点看起来离参考曲面很远,
    其实是来自曲率变化剧烈的区域的其它(未在此绘制的)切片。
\end{figure}

\subsection{ROC曲线及曲线下面积}
基于\secref{expr:match}中针对单个测试图像的曲面匹配方法,
在本小节中我们给出其ROC曲线绘制的方法,以及多个测试图像的整体评分方法。

假设有$M$个测试图像,固定特征距离阈值$\theta$,对每个测试图像,
仅保留$\theta$以下的匹配点,则此时可以得到$(a_1^{(\theta)},
t_1^{(\theta)}),\cdots,(a_M^{(\theta)}, t_M^{(\theta)})$共$M$个二元组,
$a_i^{(\theta)}$表示所有训练图像在第$i$个测试图像上
距离不超过$\theta$的匹配点中成功匹配的比例,即$\theta$限制下的匹配精确度;
$t_i^{(\theta)}$表示这些匹配点占所有训练图像的参考曲面上选中的点的比例,
沿用上节记号,则这些选中的点的总数为$NT$,
$t_i^{(\theta)}$就是$\theta$限制下总匹配点数与$NT$之商。
记$a^{(\theta)}$、$s^{(\theta)}$分别为
$a_1^{(\theta)},\cdots,a_M^{(\theta)}$的均值和方差,
$t^{(\theta)}$为$t_1^{(\theta)},\cdots,t_M^{(\theta)}$的均值,
显然$t^{(\theta)}$随着$\theta$增加是单调不下降的。
遍历$\theta$取值,
将所有$(t^{(\theta)}, a^{(\theta)})$对应的平面点连接起来,
就得到了ROC曲线,
同时可以作出$a^{(\theta)} \pm s^{(\theta)}$的对应区域来反应评测的准确度。


\section{实验配置}

\subsection{实验环境}

\subsection{训练数据}

\subsection{模型参数设定}


\section{实验结果}
曲线,表格


% vim: filetype=tex foldmethod=marker foldmarker=f{{{,f}}}


