% $File: discuss.tex
% $Date: Mon Jun 22 17:19:40 2015 +0800
% $Author: jiakai <jia.kai66@gmail.com>

\chapter{总结与展望\label{chap:discuss}}

\section{本文工作总结}
在本文中,我们对基于非监督学习的医学影像特征提取方法进行了研究。

我们以层叠卷积ISA作为基准方法,
在\chapref{ISA}中介绍了其基本数学原理和我们数据并行实现的策略,
同时将相关代码开源。

随后,针对ISA方法固有的缺陷,
我们在\chapref{CNN}中提出需要人工引导特征提取器学习我们想要的不变性、
并让特征同时具有鲁棒性和区分度的总体思想。
为此,我们设计了多分类输出和度量学习两种损失函数,
并将其连接在我们设计的深度卷积神经网络之后。
我们将所需的不变性具体局限在对仿射变换和Gamma校正的不变性,
并针对3D扫描影像提出了对变换函数的参数进行均匀采样的方法。

在\chapref{expr}中,为了测试各种特征提取器的性能,我们设计了一种新的评测标准,
其基本思想是在人工标定的特定器官的分割掩膜上,计算出可以精确定义的边界,
并通过计算训练图像上的边界点在测试图像上的匹配精度来反映特征性能;
这种方法的优点是不需要涉及具体的医学影像处理任务,
避免了其它更复杂的方法影响对特征性能的评测,
同时也只依赖于器官分割标注而无需具体的点对应标注,操作简便。

在我们设计的评测标准上,我们比较了各种算法在一些参数组合下的性能,
并在\secref{expr:discuss}中给出了设计高性能特征提取器所需技巧的一些初步结论。

\section{未来工作展望}
本文对在医学影像上基于非监督学习的特征提取方法进行了初步研究,
然而还是有很多问题没有研究透彻,对于该领域未来的工作,可以进一步探索以下方面:

\paragraph{不同数据上的普适性}
本文中我们仅仅以SLIVER07的腹腔CT扫描数据为例进行了研究,
并未探索我们的结论在MRI等其它来源的数据、
或者其它人体部位的扫描数据、
或者对一些物体(如行李安检)的扫描数据上是否仍然成立。
另一方面,我们也不知道在某个数据集上训练的特征提取器,
在别的数据集上是否也可以直接取得好效果。

\paragraph{网络设计与训练}
由于项目时间和资源有限,我们没有在深度卷积神经网络的设计上进行优化尝试,
而是直接采取了按我们直觉选择的初始设计。网络层数、卷积核大小、通道数、
网络拓扑结构、激活函数的选择、输入大小、特征维度
等各种因素都可能对最终性能造成影响。训练时超参数的调节、
数据增广的程度、网络蒸馏\cite{hinton2015distilling}、
度量学习时的难样本挖掘等训练方法,
也都可能极大地改变最终结果。这方面的最优选择值得进一步探索。

\paragraph{与其它方法的比较}
本文中我们仅仅对深度卷积神经网络和层叠卷积ISA进行了比较,
在未来的工作中也应该与受限玻尔兹曼机、自动编码器、
3D-SIFT等其它方法进行比较,才能对特征提取有更全面深入的理解。

\paragraph{评测标准的可靠性}
在本文中我们提出了新的评测标准,并通过人工观察数据对其有效性进行了大致确认,
但这方面需要更系统的研究。一种可能的研究思路是,
把各种特征接入分割、匹配的各种后期任务,对其在各种任务上的表现进行统计,
并对统计结果和本评测标准给出的排名间的相关性进行考察。

% vim: filetype=tex foldmethod=marker foldmarker=f{{{,f}}}

